\chapter*{Introduzione}
\markboth{Introduzione}{}
L'azienda Florence Consulting Group presso la quale ho svolto il tirocinio ha sviluppato per un proprio cliente un'applicazione Android. In questo lavoro sono riportate le procedure svolte al fine di effettuare un assessment di sicurezza di questa applicazione ed i risultati ottenuti. Sono qui contenute tutte le informazioni inerenti il processo di discovery delle vulnerabilità, l'attività di \ac{PT} e le possibili contromisure da adottare.  

L'attività di PT è stata eseguita con l'approccio \emph{black box}; infatti l'unica risorsa a disposizione è l'apk originale dell'applicazione (fornito direttamente dagli sviluppatori) ed è stato simulato il comportamento di un attaccante entratone in possesso. Sono stati utilizzati numerosi strumenti automatici che verranno dettagliati nelle successive sezioni.

Oltre all'esecuzione di scansioni statiche direttamente sul pacchetto apk, l'applicazione è stata installata su uno smartphone \emph{Nexus 4} dotato di sistema operativo \emph{Android} $5.1.1$ (Lollipop) per poter eseguire analisi dinamiche sul suo comportamento a run-time.

Infine, è stata creata una versione malevola dell'applicazione, indistinguibile dall'originale, contenente una backdoor alla quale è possibile collegarsi da remoto. Questa applicazione, una volta installata e lanciata, permette ad un attaccante esterno di connettersi al dispositivo della vittima e di acquisirne il controllo totale in maniera completamente trasparente all'utente legittimo.

Il testo è organizzato come segue:

\begin{itemize}
	\item Nelle sezioni $1$ e $2$ vengono rispettivamente presentati gli strumenti utilizzati per effettuare le scansioni statiche e dinamiche.
	\item Nella sezione $3$ vengono presentati e commentati i risultati delle scansioni effettuate.
	\item Nella sezione $4$ viene presentata l'applicazione contraffatta, il suo utilizzo e le sue potenzialità.
	\item In appendice è riportata la lista "\emph{mobile top $10$ risks 2016}" stilata da \ac{OWASP}, il riferimento principale per questo lavoro.
\end{itemize}